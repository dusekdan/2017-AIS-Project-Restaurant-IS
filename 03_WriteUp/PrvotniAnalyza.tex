\documentclass[a4paper,10pt]{article}
\usepackage[utf8]{inputenc}
\usepackage{graphicx}
\usepackage{pdfpages}
\usepackage{afterpage}
\usepackage{scrextend}
\usepackage{pdflscape}
\usepackage{caption}
\usepackage{array}
\usepackage{enumitem}
\usepackage[utf8]{inputenc}
\usepackage[czech]{babel}
\usepackage{adjustbox}
\usepackage[left=2.5cm,right=2.5cm,top=2.5cm,bottom=0.5cm]{geometry}
\title{AIS - Restaurace}
\begin{document}


% Helper commands
\providecommand{\uv}[1]{\quotedblbase #1\textquotedblleft}

\renewcommand{\figurename}{Obrázek}
\newcommand\textbox[1]{
	\parbox{.5\textwidth}{#1}
}

% === FRONT PAGE ===
\thispagestyle{plain}
	\newgeometry{left=2cm,right=2cm,top=1.5cm}
	\pagenumbering{gobble}
	\begin{center}
		\Huge
		VYSOKÉ UČENÍ TECHNICKÉ V BRNĚ \\
			\vspace{\stretch{0.150}}


	\includegraphics[width=\textwidth]{resources/fit-logo.pdf}
			\vspace{\stretch{0.300}}



		\Large{Projekt do předmětu AIS \\ ~ \\}
		
		\LARGE
		\textsc{Prvotní analýza a plán projektu}
			\vspace{\stretch{0.618}}

	\end{center}

	\noindent \textbox{18. října, 2017} \textbox{\hfill \textbf{Autoři:}  Daniel Dušek ~~(xdusek21) ~~~}
	\noindent \textbox{\hfill} \textbox{\hfill Filip Kalous ~~~(xkalou03) ~~~}
	\noindent \textbox{\hfill} \textbox{\hfill Anna Popková (xpopko00)~~~}

	\clearpage
	\restoregeometry

\newpage

\normalsize

\newpage
\thispagestyle{empty}
\section*{Neformální specifikace projektu}
Středně velká restaurace potřebuje informační systém, který usnadní komunikaci a~synchronizaci mezi personálem kuchyně, obsluhy, kanceláří a~zákazníky. Zejména pak pro řešení rezervací stolů, salónku, přijatých a~vyhotovených objednávek jídla.

Restaurace sestává ze čtyř logických částí. Kuchyně, část ve které se pohybuje obsluha, místa v~restauraci a~kancelář. Systém by měl umožnit pracovníkům obsluhy předat informaci o~objednaných jídlech do kuchyně a~následně pak pracovníkům kuchyně zobrazit objednaná jídla a~upravovat stav jejich vyhotovení.

Běžný zákazník restaurace přijde do styku pouze s~webovou aplikací, která systém rozšiřuje, a~umožňuje zarezervovat si konkrétní místo u~stolu, stůl, popřípadě celý salónek v~uživatelem zvoleném termínu. Prostřednictvím rezervačního rozhraní si může také zákazník zobrazit kdy jsou a~nejsou místa k~rezervaci volná. Uživatel dále může svou rezervaci stornovat. Nezávisle k~rezervaci může také poslat zpětnou vazbu majiteli restaurace prostřednictvím formuláře dostupného v~na systému nezávislé webové aplikaci.

Pracovníci obsluhy si mohou zobrazovat existující rezervace a~stav stolů a~salónků. Dále mohou prostřednictvím systému označovat stoly jako volné a~obsazené. Obsluha dále může zadávat do systému objednávky pro kuchyň a~sledovat stav těchto objednávek.

Kuchyňský personál může zobrazovat objednaná jídla a~měnit jejich stav podle toho, jak se daří jejich vyhotovení. Kuchyňský personál může žádat kancelář o~dozásobení konkrétním typem suroviny, která v~kuchyni chybí. Dále může kuchyňský personál editovat denní menu restaurace.

Kancelářský personál potvrzuje, zamítá a~ruší prostřednictvím systému rezervace, které vytvořili zákazníci skrze webovou aplikaci. Systém si tyto vytvořené rezervace v~pravidelných intervalech stahuje.

Vedení restaurace má k~dispozici jakoukoliv operaci, kterou má k~dispozici libovolný aktér v~systému.


\section*{Prvotní analýza požadavků}

Dle neformální specifikace budou systém používat dvě hlavní skupiny aktérů:
\begin{enumerate}
\item \textbf{Aktér Zákazník}:
Jeho cílem je zprostředkovaně pomocí webové aplikace uskutečnit rezervaci v~restauraci, prohlídnout si aktuální menu a~případně aktuální obsazenost restaurace.
\item \textbf{Aktér Personál}:
Jedná se o~abstraktního aktéra spojeného s~obecnými akcemi jako je zobrazení rezervací a~aktuálníh obsazenosti restaurace. V~kontextu vztahu generalizace-specializace jsou specializací této skupiny následující potomci:
\begin{enumerate}
\item  \textbf{Aktér Kancelář}, který spravuje rezervace vytvořené aktérem Zákazníkem.
\item  \textbf{Aktér Obslužný personál}, který si tyto rezervace může prohlížet, a~který zadává do systému objednávky zákazníků sedících přímo v~restauraci.
\item  \textbf{Aktér Kuchyňský personál}, který aktuální objednávky spravuje a~také může podat požadavek na dozásobení chybějící suroviny v~kuchyni.
\end{enumerate}
Nakonec v systému vystupuje speciální \textbf{aktér Vedení}, který má přístup k~jakékoliv akci, ke které mají přístup aktéři spadající pod skupinu Personál.
\end{enumerate}

Vztahy a případy užití jednotlivých aktérů jsou zobrazeny v následujícím diagramu:

Z~implementačního hlediska bude systém pro restauraci realizován jako samostatná aplikace s přístupem k privátní databázi, ke které bude možné získat přístup pouze z~vnitřní sítě restaurace. Související webová aplikace bude informace o~rezervacích, aktuální obsazenosti restaurace a~menu ukládat do oddělené databáze. Tyto informace budou do webové aplikace zasílány zevnitř informačního systému při každé jejich změně. Obráceně pak, informační systém bude získávat v pravidelných intervalech informace o~nových rezervacích a~zobrazovat je pracovníkům v~kanceláři ke zpracování.

\section*{Plán projektu}

Vzhledem ke komplexnosti projektu jsme se rozhodli rozdělit vývoj tohoto systému do tří iterací, kde v~každé iteraci bude přidána určitá klíčová funkcionalita.

V~první iteraci bude systém umět obsloužit požadavky na rezervace a~s~tím spojené operace. V~této fázi v~systému vystupují pouze základní aktéři: Zákazník, Personál a~Vedení.
Případy užití pokryté v~této iteraci zobrazuje obrázek~1.

Ve druhé iteraci budou rozšířeny možnosti rezervací a~dále bude přidána možnost správy objednávek. V~této fázi již v~systému vystupují všichni aktéři.
Funkcionalita pokrytá v~této iteraci je zachycena v diagramu na obrázku~2.

Ve třetí (závěrečné) iteraci jsou vylepšeny možnosti organizace prostoru a~služeb restaurace tak, že bude systém odpovídat diagramu na obrázku~3.

%\includegraphics[landscape]{resources/iteration01.pdf}
\begin{center}
\afterpage{
	\pagenumbering{gobble}
	\includepdf[offset=3cm 5cm,pagecommand={
		\thispagestyle{empty}
		\null\vfill
		\captionof{figure}{Diagram použití v první iteraci}}]{resources/iteration01.pdf}
	\label{fig:iteration01}
	\clearpage
	\restoregeometry
}
\end{center}


\begin{landscape}	
	\afterpage{
		
		\pagenumbering{gobble}

		\includepdf[angle=90,scale=0.80,
			pagecommand={
				\thispagestyle{plain}
				\null\vfill
				\captionof{figure}{Diagram použití v druhé iteraci}}]{resources/iteration02.pdf}
		\label{fig:iteration02}
	
		\clearpage
		\restoregeometry
	}
\end{landscape}


\begin{landscape}	
	\afterpage{
			
			\pagenumbering{gobble}

			\includepdf[angle=90,scale=0.70,
				pagecommand={
					\thispagestyle{plain}
					\null\vfill
					\captionof{figure}{Diagram použití v třetí iteraci}}]{resources/iteration03.pdf}
			\label{fig:iteration03}
		
			\clearpage
			\restoregeometry
		}
\end{landscape}


\newpage
%=====================================================================================
%
%                    1. ITERACE 
%
%
%======================================================================================
% MODELY 1. ITERACE %
% === FRONT PAGE ===
\thispagestyle{plain}
	\newgeometry{left=2cm,right=2cm,top=1.5cm}
	\pagenumbering{gobble}
	\begin{center}
		\Huge
		VYSOKÉ UČENÍ TECHNICKÉ V BRNĚ \\
			\vspace{\stretch{0.150}}


	\includegraphics[width=\textwidth]{resources/fit-logo.pdf}
			\vspace{\stretch{0.300}}



		\Large{Projekt do předmětu AIS \\ ~ \\}
		
		\LARGE
		\textsc{Modely --- 1. iterace}
			\vspace{\stretch{0.618}}

	\end{center}

	\noindent \textbox{20. listopadu, 2017} \textbox{\hfill \textbf{Autoři:}  Daniel Dušek ~~(xdusek21) ~~~}
	\noindent \textbox{\hfill} \textbox{\hfill Filip Kalous ~~~(xkalou03) ~~~}
	\noindent \textbox{\hfill} \textbox{\hfill Anna Popková (xpopko00)~~~}

	\clearpage
	\restoregeometry

\newpage

\normalsize

\newpage
\section*{Diagram případu použití}
\begin{figure}[h!]
\begin{center}
\includegraphics[scale=0.75]{resources/iteration01.pdf}
\captionof{figure}{Diagram použití v první iteraci}
\label{fig:iteration01}
\end{center}
\end{figure}

\newpage
\section*{Specifikace případů použití}
\textbf{Případ užití \uv{Vytvořit rezervaci}}

%%%%%%%%%%%%%%%%%%%%%%%
% Vytvoření rezervace %  
%%%%%%%%%%%%%%%%%%%%%%%
\begin{table}[ht!]
{\renewcommand{\arraystretch}{1.3}
\begin{tabular}{| r | p{12cm} |}
	\hline
	ID: & 1 \\
    \hline
    Název: & \textbf{Vytvořit rezervaci} \\
    \hline
    Vytvořeno: & Filip Kalous, Daniel Dušek, Anna Popková \\
    \hline
    Popis: & Uživatel vytvoří rezervaci \\
    \hline
    Primární aktéři: & Zákazník \\
    \hline
    Sekundární aktéři: & Systém \\
    \hline
    Předpoklady: & Žádné \\
    \hline
    Následné podmínky: & 
    \begin{minipage}[t]{0.75\textwidth}
    	\begin{enumerate}[nosep,after=\strut]
    		\item V systému je vytvořena zákazníkova rezervace.
    	\end{enumerate}
  	\end{minipage} \\
	\hline
    Akce pro spuštění: & Uživatel zvolí událost \uv{Vytvořit rezervaci} \\
    \hline
    Hlavní tok: & 
    \begin{minipage}[t]{0.75\textwidth}
    	\begin{enumerate}[nosep,after=\strut]
            \item Uživatel vyplní potřebné informace pro rezervaci
            \item Uživatel zvolí, zda chce salónek nebo konkrétní stůl
            \item Uživatel vyplní informace o své osobě
            \item Uživatel zvolí událost \uv{Potvrdit rezervaci}
    	\end{enumerate}
  	\end{minipage} \\
    \hline
    Alternativní toky: & 
    \begin{minipage}[t]{0.75\textwidth}
    	\begin{enumerate}[nosep,after=\strut]
            \item Uživatel vyplní potřebné informace pro rezervaci, ale systém mu sdělí, že restaurace je plně obsazena.
    	\end{enumerate}
  	\end{minipage} \\
    \hline
    Výjimky: & 
    \begin{minipage}[t]{0.75\textwidth}
    	\begin{enumerate}[nosep,after=\strut]
    		\item V době potvrzení rezervace je požadované místo již zabrané.
            \item Selhání načtení plánku restaurace
            \item Selhání systému
    	\end{enumerate}
  	\end{minipage} \\
    \hline
    Frekvence: & Často \\
    \hline
    Speciální požadavky: & 
    \begin{minipage}[t]{0.75\textwidth}
    	\begin{enumerate}[nosep,after=\strut]
    		\item Žádné
    	\end{enumerate}
  	\end{minipage} \\
    \hline

\end{tabular}}
\caption{Specifikace případu užití \uv{Vytvořit rezervaci}}
\label{table:1}
\end{table}


%%%%%%%%%%%%%%%%%%%%%%%%%%%%%%%%%%%
% Výjimka 1 - Vytvoření rezervace %  
%%%%%%%%%%%%%%%%%%%%%%%%%%%%%%%%%%%

\par{\textbf{Výjimky případu užití \uv{Vytvořit rezervaci}}}
\begin{table}[ht!]
{\renewcommand{\arraystretch}{1.3}
\begin{tabular}{| r | p{12cm} |}
	\hline
	ID: & 1.E.1 \\
    \hline
    Název: & \textbf{Vytvořit rezervaci: V době potvrzení rezervace je požadované místo již zabrané.} \\
    \hline
    Vytvořeno: & Filip Kalous, Daniel Dušek, Anna Popková \\
    \hline
    Popis: & Rezervace není potvrzena z důvodu zabraného místa. \\
    \hline
    Primární aktéři: & Systém \\
    \hline
    Sekundární aktéři: &  \\
    \hline
    Předpoklady: & V době potvrzování rezervace systém zjistí, že vybraná místa jsou již zarezervována.  \\
    \hline
    Následné podmínky: & 
	\begin{minipage}[t]{0.75\textwidth}
 		\begin{enumerate}[nosep,after=\strut]
 			\item Uživatel se nachází na stránce rezervačního formuláře.
 			\item Uživateli je zobrazena zpráva o neúspěchu vytvoření rezervace.
 		\end{enumerate}
    \end{minipage} \\
	\hline
    Akce pro spuštění: & Selhání rezervace. \\
    \hline
    Tok: & 
    \begin{minipage}[t]{0.75\textwidth}
    	\begin{enumerate}[nosep,after=\strut]
            \item Přesměrování uživatele zpět na rezervační formulář.
            \item Systém zobrazi chybovou hlášku o neúspěchu vytvoření rezervace.
    	\end{enumerate}
    \end{minipage} \\
    \hline
    Frekvence: & Zřídka \\
    \hline

\end{tabular}}
\caption{Vytvořit rezervaci: V době potvrzení rezervace je požadované místo již zabrané.}
\label{table:2}
\end{table}

%%%%%%%%%%%%%%%%%%%%%%%%%%%%%%%%%%%
% Výjimka 2 - Vytvoření rezervace %  
%%%%%%%%%%%%%%%%%%%%%%%%%%%%%%%%%%%
\begin{center}
\begin{table}[ht!]
{\renewcommand{\arraystretch}{1.3}
\begin{tabular}{| r | p{12cm} |}
	\hline
	ID: & 1.E.2 \\
    \hline
    Název: & \textbf{Vytvořit rezervaci: Selhání načtení plánku restaurace} \\
    \hline
    Vytvořeno: & Filip Kalous, Daniel Dušek, Anna Popková \\
    \hline
    Popis: & Systém nemůže načíst plánek restaurace. \\
    \hline
    Primární aktéři: & Systém \\
    \hline
    Sekundární aktéři: &  \\
    \hline
    Předpoklady: & 
    \begin{minipage}[t]{0.75\textwidth}
    	\begin{enumerate}[nosep,after=\strut]
    		\item Aplikace se snaží zobrazit neexistujíci (nesprávná) data.
    	\end{enumerate}
  	\end{minipage} \\
    \hline
    Následné podmínky: & 
    \begin{minipage}[t]{0.75\textwidth}
    	\begin{enumerate}[nosep,after=\strut]
    		\item Zobrazena chybová hláška
    	\end{enumerate}
  	\end{minipage} \\
	\hline
    Akce pro spuštění: & Selhání načtení plánku restaurace. \\
    \hline
    Tok: & 
    \begin{minipage}[t]{0.75\textwidth}
    	\begin{enumerate}[nosep,after=\strut]
        	\item Systém zobrazí chybvou hlášku.
            \item Systém se pokusí o znovunačtení plánku restaurace.
    	\end{enumerate}
  	\end{minipage} \\
    \hline
    Frekvence: & Zřídka \\
    \hline

\end{tabular}}
\caption{Vytvořit rezervaci: Selhání načtení plánku restaurace}
\label{table:3}
\end{table}
\end{center}
\newpage 
%%%%%%%%%%%%%%%%%%%%%%%%%%%%%%%%%%%
% Výjimka 3 - Vytvoření rezervace %  
%%%%%%%%%%%%%%%%%%%%%%%%%%%%%%%%%%%
\begin{center}
\begin{table}[ht!]
{\renewcommand{\arraystretch}{1.3}
\begin{tabular}{| r | p{12cm} |}
	\hline
	ID: & 1.E.3 \\
    \hline
    Název: & \textbf{Vytvořit rezervaci: Selhání systému} \\
    \hline
    Vytvořeno: & Filip Kalous, Daniel Dušek, Anna Popková \\
    \hline
    Popis: & Systém nedokáže pokračovat v případu. \\
    \hline
    Primární aktéři: & Systém \\
    \hline
    Sekundární aktéři: &  \\
    \hline
    Předpoklady: & 
    \begin{minipage}[t]{0.75\textwidth}
    	\begin{enumerate}[nosep,after=\strut]
    		\item Systém neprovedl korektně některý z kroku hlavního toku případu užití.
            \item Systém nespadl.
    	\end{enumerate}
  	\end{minipage} \\
    \hline
    Následné podmínky: & 
    \begin{minipage}[t]{0.75\textwidth}
    	\begin{enumerate}[nosep,after=\strut]
    		\item Systém nevytvořil rezervaci.
    	\end{enumerate}
  	\end{minipage} \\
	\hline
    Akce pro spuštění: & Selhání systému v libovolném místě toku případu \uv{Vytvořit rezervaci}. \\
    \hline
    Tok: & 
    \begin{minipage}[t]{0.75\textwidth}
    	\begin{enumerate}[nosep,after=\strut]
            \item Systém informuje uživatele chybovou hláškou o selhání.
            \item Systém přesměruje uživatele zpět na začátek rezervačního formuláře.
    	\end{enumerate}
  	\end{minipage} \\
    \hline
    Frekvence: & Zřídka \\
    \hline

\end{tabular}}
\caption{Vytvořit rezervaci: Selhání systému}
\label{table:4}
\end{table}
\end{center}

\newpage
%%%%%%%%%%%%%%%%%%%%%%%
% Zamítnout rezervaci %
%%%%%%%%%%%%%%%%%%%%%%%
\textbf{Případ užití \uv{Zamítnout rezervaci}}
\begin{center}
\begin{table}[ht!]
{\renewcommand{\arraystretch}{1.3}
\begin{tabular}{| r | p{12cm} |}
	\hline
	ID: & 2 \\
    \hline
    Název: & \textbf{Zamítnout rezervaci} \\
    \hline
    Vytvořeno: & Filip Kalous, Daniel Dušek, Anna Popková \\
    \hline
    Popis: & Kancelář zamítne uživatelem vytvořenou rezervaci. \\
    \hline
    Primární aktéři: & Kancelář \\
    \hline
    Sekundární aktéři: & Systém \\
    \hline
    Předpoklady: & Existuje rezervace, kterou není možné potvrdit. \\
    \hline
    Následné podmínky: & 
    \begin{minipage}[t]{0.75\textwidth}
    	\begin{enumerate}[nosep,after=\strut]
    		\item Rezervace je zamítnuta.
            \item Zákazník je informován o zamítnutí rezervace.
    	\end{enumerate}
  	\end{minipage} \\
	\hline
    Akce pro spuštění: & Kancelář zvolí akci \uv{Zamítnout rezervaci} \\
    \hline
    Hlavní tok: & 
    \begin{minipage}[t]{0.75\textwidth}
    	\begin{enumerate}[nosep,after=\strut]
    		\item Kancelář vybere rezervaci, kterou chce zamítnout.
            \item Kancelář popíše důvod zamítnutí rezervace.
            \item Kancelář potvrdí zamítnutí rezervace.
            \item Systém zašle informaci zákazníkovi o zamítnutí rezervace.
    	\end{enumerate}
  	\end{minipage} \\
    \hline
    Alternativní toky: & \\
    \hline
    Výjimky: & 
    \begin{minipage}[t]{0.75\textwidth}
    	\begin{enumerate}[nosep,after=\strut]
    		\item Selhání systému
    	\end{enumerate}
  	\end{minipage} \\
    \hline
    Frekvence: & Velmi často \\
    \hline
    Speciální požadavky: & 
    \begin{minipage}[t]{0.75\textwidth}
    	\begin{enumerate}[nosep,after=\strut]
    		\item Žádné
    	\end{enumerate}
  	\end{minipage} \\
    \hline

\end{tabular}}
\caption{Specifikace případu užití \uv{Zamítnout rezervaci}}
\label{table:5}
\end{table}
\end{center}

\newpage
%%%%%%%%%%%%%%%%%%%%%%%%%%%%%%%%%%%
% Výjimka 1 - Zamítnutí rezervace %  
%%%%%%%%%%%%%%%%%%%%%%%%%%%%%%%%%%%
\textbf{Výjimka případu užití \textbf{Zamítnout rezervaci}}

\begin{table}[ht!]
{\renewcommand{\arraystretch}{1.3}
\begin{tabular}{| r | p{12cm} |}
	\hline
	ID: & 2.E.1 \\
    \hline
    Název: & \textbf{Zamítnutí rezervace: Selhání systému} \\
    \hline
    Vytvořeno: & Filip Kalous, Daniel Dušek, Anna Popková \\
    \hline
    Popis: & Systém nedokáže pokračovat v případu. \\
    \hline
    Primární aktéři: & Systém \\
    \hline
    Sekundární aktéři: &  \\
    \hline
    Předpoklady: & 
    \begin{minipage}[t]{0.75\textwidth}
    	\begin{enumerate}[nosep,after=\strut]
    		\item Systém neprovedl korektně některý z kroku hlavního toku případu užití.
            \item Systém nespadl.
    	\end{enumerate}
  	\end{minipage} \\
    \hline
    Následné podmínky: & 
    \begin{minipage}[t]{0.75\textwidth}
    	\begin{enumerate}[nosep,after=\strut]
    		\item Systém nepotvrdil zamítnutí rezervace.
            \item Je zobrazena chybová hláška.
    	\end{enumerate}
  	\end{minipage} \\
	\hline
    Akce pro spuštění: & Selhání systému v libovolném místě toku případu \uv{Vytvořit rezervaci} \\
    \hline
    Tok: & 
    \begin{minipage}[t]{0.75\textwidth}
    	\begin{enumerate}[nosep,after=\strut]
            \item Systém informuje uživatele chybovou hláškou o selhání.
            \item Systém přesměruje Vedení zpět na první krok zamítnutí rezervace.
    	\end{enumerate}
  	\end{minipage} \\
    \hline
    Frekvence: & Zřídka \\
    \hline

\end{tabular}}
\caption{Zamítnutí rezervace: Selhání systému}
\label{table:6}
\end{table}


\newpage
\textbf{Případ použití: \uv{Zrušit rezervaci}}
\begin{center}
\begin{table}[ht!]
{\renewcommand{\arraystretch}{1.3}
\begin{tabular}{| r | p{12cm} |}
	\hline
	ID: & 3 \\
    \hline
    Název: & \textbf{Zrušit rezervaci} \\
    \hline
    Vytvořeno: & Filip Kalous, Daniel Dušek, Anna Popková \\
    \hline
    Popis: & Zruší existující rezervaci \\
    \hline
    Primární aktéři: & Vedení \\
    \hline
    Sekundární aktéři: &  Zákazník \\
    \hline
    Předpoklady: & Existuje zákazníkem vytvořená a vedením potvrzená rezervace \\
    \hline
    Následné podmínky: & 
    \begin{minipage}[t]{0.75\textwidth}
    	\begin{enumerate}[nosep,after=\strut]
    		\item Rezervace je zrušena. 
            \item Aktér \uv{Zákazník} je o tom informován.
    	\end{enumerate}
  	\end{minipage} \\
    \hline
        Akce pro spuštění: & 
    	Aktér \uv{Vedení} restaurace v seznamu potvrzených rezervací klikne na tlačítko \uv{Zrušit rezervaci} u některého z existujících záznamů \\
    \hline
    Hlavní tok: & 
    \begin{minipage}[t]{0.75\textwidth}
    	\begin{enumerate}[nosep,after=\strut]
    		\item Systém zobrazí seznam potvrzených rezervací.
            \item Uživatel v seznamu potvrzených rezervací klikne na tlačítko zrušit rezervaci.
            \item Uživatel potvrdí, že zrušení rezervace.
            \item Systém zruší rezervaci a pošle zprávu na adresu zákazníka o zrušení rezervace.
    	\end{enumerate}
  	\end{minipage} \\
    \hline
    Alternativní toky: & \\ 
    \hline
    Výjimky: & 
    \begin{minipage}[t]{0.75\textwidth}
    	\begin{enumerate}[nosep,after=\strut]
    		\item Zákazník zruší rezervaci dříve než Vedení potvrdí, že chce rezervaci zrušit.
            \item Rušená rezervace již neexistuje. 
    	\end{enumerate}
  	\end{minipage} \\
    \hline
    Frekvence: & Málo časté \\
    \hline
    Speciální požadavky: & 
    \begin{minipage}[t]{0.75\textwidth}
    	\begin{enumerate}[nosep,after=\strut]
    		\item Vedení obdrželo prioritní požadavek na rezervaci konkrétního místa prostřednictvím kanálu mimo IS.
            \item Nemožnost použití místa, které je předmětem existující rezervace (například fyzické poškození).
    	\end{enumerate}
  	\end{minipage} \\
    \hline
\end{tabular}}
\caption{Specifikace případu užití \uv{Zrušit rezervaci}}
\label{table:7}
\end{table}
\end{center}


%%%%%%%%%%%%%%%%%%%%%%%%%%%%%%%%%
% Výjimka 1 - Zrušení rezervace %  
%%%%%%%%%%%%%%%%%%%%%%%%%%%%%%%%%
\newpage
\textbf{Výjimky případu užití \uv{Zrušit rezervaci}}
\begin{table}[ht!]
{\renewcommand{\arraystretch}{1.3}
\begin{tabular}{| r | p{12cm} |}
	\hline
	ID: & 3.E.1 \\
    \hline
    Název: & \textbf{Zrušit rezervaci: Zákazník zruší rezervaci dříve než Vedení potvrdí, že chce rezervaci zrušit.} \\
    \hline
    Vytvořeno: & Filip Kalous, Daniel Dušek, Anna Popková \\
    \hline
    Popis: & Rezervace kterou Vedení chce zrušit již byla zrušena. \\
    \hline
    Primární aktéři: & Vedení, Zákazník\\
    \hline
    Sekundární aktéři: &  Systém \\
    \hline
    Předpoklady: & Existuje potvrzená rezervace, kterou zákazník zrušil krátce před tím, než se rezervaci rozhodlo zrušit Vedení.  \\
    \hline
    Následné podmínky: & 
	\begin{minipage}[t]{0.75\textwidth}
 		\begin{enumerate}[nosep,after=\strut]
 			\item Vedení je informováno o tom, že rezervace již byla zrušena ze strany zákazníka.
 		\end{enumerate}
    \end{minipage} \\
	\hline
    Akce pro spuštění: & Kliknutí na tlačítko \uv{Zrušit rezervaci} v čase kdy rezervace již byla zrušena ze strany zákazníka.\\
    \hline
    Tok: & 
    \begin{minipage}[t]{0.75\textwidth}
    	\begin{enumerate}[nosep,after=\strut]
            \item Zobrazení informační hlášky o tom, že rezervace již byla zrušena ze strany zákazníka.
    	\end{enumerate}
    \end{minipage} \\
    \hline
    Frekvence: & Zřídka \\
    \hline

\end{tabular}}
\caption{Zrušit rezervaci: Zákazník zruší rezervaci dříve než Vedení potvrdí, že chce rezervaci zrušit.}
\label{table:8}
\end{table}

%%%%%%%%%%%%%%%%%%%%%%%%%%%%%%%%%
% Výjimka 2 - Zrušení rezervace %  
%%%%%%%%%%%%%%%%%%%%%%%%%%%%%%%%%
\begin{table}[ht!]
{\renewcommand{\arraystretch}{1.3}
\begin{tabular}{| r | p{12cm} |}
	\hline
	ID: & 3.E.2 \\
    \hline
    Název: & \textbf{Zrušit rezervaci: Rušená rezervace již neexistuje} \\
    \hline
    Vytvořeno: & Filip Kalous, Daniel Dušek, Anna Popková \\
    \hline
    Popis: & Rezervace kterou Vedení chce zrušit již v systému neexistuje. \\
    \hline
    Primární aktéři: & Vedení\\
    \hline
    Sekundární aktéři: &  Systém \\
    \hline
    Předpoklady: & V uživatelském rozhraní systému je zobrazena potvrzená rezervace, která již byla zrušena (například v jiném tabu prohlížeče). \\
    \hline
    Následné podmínky: & 
	\begin{minipage}[t]{0.75\textwidth}
 		\begin{enumerate}[nosep,after=\strut]
 			\item Vedení je informováno o tom, že rezervace již neexistuje.
            \item Neexistující položka zmizí z uživatelského rozhraní.
 		\end{enumerate}
    \end{minipage} \\
	\hline
    Akce pro spuštění: & Kliknutí na tlačítko \uv{Zrušit rezervaci} v čase kdy rezervace již v systému neexistuje.\\
    \hline
    Tok: & 
    \begin{minipage}[t]{0.75\textwidth}
    	\begin{enumerate}[nosep,after=\strut]
            \item Zobrazení informační hlášky o tom, že rezervace již v systému neexistuje.
            \item Odstranění neexistující rezervace z uživatelského rozhraní.
            \item Aktualizace uživatelského rozhraní pro odstranění dalších potenciálně neexistujících rezervací.
    	\end{enumerate}
    \end{minipage} \\
    \hline
    Frekvence: & Málo často \\
    \hline

\end{tabular}}
\caption{Zrušit rezervaci: Zákazník zruší rezervaci dříve než Vedení potvrdí, že chce rezervaci zrušit.}
\label{table:9}
\end{table}


\newpage
%=====================================================================================
%
%                    POSLEDNÍ ITERACE 
%
%
%======================================================================================
% MODELY 1. ITERACE %
% === FRONT PAGE ===
\thispagestyle{plain}
	\newgeometry{left=2cm,right=2cm,top=1.5cm}
	\pagenumbering{gobble}
	\begin{center}
		\Huge
		VYSOKÉ UČENÍ TECHNICKÉ V BRNĚ \\
			\vspace{\stretch{0.150}}


	\includegraphics[width=\textwidth]{resources/fit-logo.pdf}
			\vspace{\stretch{0.300}}



		\Large{Projekt do předmětu AIS \\ ~ \\}
		
		\LARGE
		\textsc{Výsledné modely}
			\vspace{\stretch{0.618}}

	\end{center}

	\noindent \textbox{5. prosince, 2017} \textbox{\hfill \textbf{Autoři:}  Daniel Dušek ~~(xdusek21) ~~~}
	\noindent \textbox{\hfill} \textbox{\hfill Filip Kalous ~~~(xkalou03) ~~~}
	\noindent \textbox{\hfill} \textbox{\hfill Anna Popková (xpopko00)~~~}

	\clearpage
	\restoregeometry

\newpage

\normalsize



%%%%%%%%%%%%%%%%%%%%%%%
% Potvrzení rezervace %  
%%%%%%%%%%%%%%%%%%%%%%%
\section*{Specifikace případů užití}
\textbf{Případ užití \uv{Potvrdit rezervaci}}
\begin{table}[ht!]
{\renewcommand{\arraystretch}{1.3}
\begin{tabular}{| r | p{12cm} |}
	\hline
	ID: & 4 \\
    \hline
    Název: & \textbf{Potvrdit rezervaci} \\
    \hline
    Vytvořeno: & Filip Kalous, Daniel Dušek, Anna Popková \\
    \hline
    Popis: & Kancelář potvrdí rezervaci vytvořenou zákazníkem \\
    \hline
    Primární aktéři: & Kancelář, Systém \\
    \hline
    Sekundární aktéři: & Zákazník \\
    \hline
    Předpoklady: & Zákazník někdy v minulosti vytvořil žádost o rezervaci, která nebyla zamítnuta systémem, ani zpracována pracovníkem Kancelář. \\
    \hline
    Následné podmínky: & 
    \begin{minipage}[t]{0.75\textwidth}
    	\begin{enumerate}[nosep,after=\strut]
    		\item V systému je potvrzena rezervace zákazníka.
            \item Zákazníkovi je na email zaslán rezervační klíč.
    	\end{enumerate}
  	\end{minipage} \\
	\hline
    Akce pro spuštění: & Kancelář klikne na tlačítko \uv{Potvrdit rezervaci} \\
    \hline
    Hlavní tok: & 
    \begin{minipage}[t]{0.75\textwidth}
    	\begin{enumerate}[nosep,after=\strut]
            \item Aktér Kancelář klikne na odkaz \uv{Potvrzovat rezervace}.
            \item Systém zobrazí uživateli seznam vytvořených rezervací, které čekají na potvrzení.
            \item Aktér Kancelář vybere ze seznamu čekajících rezervací tu, kterou chce potvrdit a klikne na tlačítko \uv{Potvrdit rezervaci}.
            \item Systém zkontroluje zda jsou splněny všechny podmínky, za kterých může být rezervace potvrzena a podmínky jsou úspěšně splněny.
            \item Systém vygeneruje unikátní identifikátor rezervace a \uv{Storno} odkaz, který zašle na email Zákazníka vytvářejícího rezervaci.
            \item Systém zobrazí informaci o potvrzení rezervace.
    	\end{enumerate}
  	\end{minipage} \\
    \hline 
    Alternativní toky: & Rezervované místo není k dispozici  \\
    \hline
    Výjimky: & 
    \begin{minipage}[t]{0.75\textwidth}
    	\begin{enumerate}[nosep,after=\strut] 
            \item Selhání systému
            \item Selhání služby odesílající email
    	\end{enumerate}
  	\end{minipage} \\
    \hline
    Frekvence: & Často \\
    \hline
    Speciální požadavky: & 
    \begin{minipage}[t]{0.75\textwidth}
    	\begin{enumerate}[nosep,after=\strut]
    		\item Žádné
    	\end{enumerate}
  	\end{minipage} \\
    \hline
\end{tabular}}
\caption{Specifikace případu užití \uv{Potvrdit rezervaci}}
\label{table:1}
\end{table}

%%%%%%%%%%%%%%%%%%%%%%%%%%%%%%%%%%%%%%%%%%%%%%%%%%%%%%%%%%
% Alternativní tok -  Rezervované místo není k dispozici %  
%%%%%%%%%%%%%%%%%%%%%%%%%%%%%%%%%%%%%%%%%%%%%%%%%%%%%%%%%%
\newpage
\textbf{Alternativní tok případu užití: \uv{Rezervované místo není k dispozici}}
\begin{table}[ht!]
{\renewcommand{\arraystretch}{1.3}
\begin{tabular}{| r | p{12cm} |}
	\hline
	ID: & 4.1 \\
    \hline
    Název: & \textbf{Potvrdit rezervaci: Rezervované místo není k dispozici.} \\
    \hline
    Vytvořeno: & Filip Kalous, Daniel Dušek, Anna Popková \\
    \hline
    Popis: & Rezervace není potvrzena z důvodu nedostupného místa. \\
    \hline
    Primární aktéři: & Kancelář, Systém \\
    \hline
    Sekundární aktéři: &  \\
    \hline
    Předpoklady: & V době potvrzování rezervace systém zjistí, že vybraná místa nejsou již k dispozici.  \\
    \hline
    Následné podmínky: & 
	\begin{minipage}[t]{0.75\textwidth}
 		\begin{enumerate}[nosep,after=\strut]
 			\item Rezervace je automaticky zamítnuta.
 			\item Zákazník je informován o tom, že jeho rezervace byla zamítnuta.
            \item Pracovník Kanceláře je informován o skutečnosti, že rezervace byla zamítnuta, včetně informace o důvodu automatického zamítnutí.
 		\end{enumerate}
    \end{minipage} \\
	\hline
    Akce pro spuštění: & Kliknutí na tlačítko Potvrdit rezervaci u rezervace, která obsahuje nedostupné místo. \\
    \hline
    Tok: & 
    \begin{minipage}[t]{0.75\textwidth}
    	\begin{enumerate}[nosep,after=\strut]
            \item Aktér Kancelář klikne na odkaz \uv{Potvrzovat rezervace}.
            \item Systém zobrazí uživateli seznam vytvořených rezervací, které čekají na potvrzení.
            \item Aktér Kancelář vybere ze seznamu čekajících rezervací tu, kterou chce potvrdit a klikne na tlačítko \uv{Potvrdit rezervaci}.
            \item Systém zkontroluje zda jsou splněny všechny podmínky, za kterých může být rezervace potvrzena a podmínky nejsou úspěšně splněny. Některé z rezervovaných míst je již obsazeno jinou rezervací.
            \item Systém vygeneruje email informující uživatele o zamítnutí jeho rezervace.
            \item Systém zobrazí informaci o automatickém zamítnutí rezervace, včetně důvodů.
    	\end{enumerate}
    \end{minipage} \\
    \hline
    Frekvence: & Málo často \\
    \hline
    Speciální požadavky: & 
    \begin{minipage}[t]{0.75\textwidth}
    	\begin{enumerate}[nosep,after=\strut]
    		\item Aktér Kancelář pracoval s více instancemi aplikace (například otevřenými taby prohlížeče) a některou ze vzájemně konfliktních rezervací potvrdil na první instanci a druhou konfliktní se pokusil potvrdit v druhé instanci.
    	\end{enumerate}
  	\end{minipage} \\
    \hline
\end{tabular}}
\caption{Potvrdit rezervaci: Rezervované místo není k dispozici}
\label{table:2}
\end{table}

%%%%%%%%%%%%%%%%%%%%%%%
% Výjimka 1&2 - Selhání systému, selhání odesílání mailu %  
%%%%%%%%%%%%%%%%%%%%%%%
\textbf{Výjimka \textit{4.E.1 Selhání systému} je popsána již v 1.E.3 z modelů případů použití v první iteraci.} \\
\newpage
\textbf{Výjimka \textit{Selhání služby odesílající email}}:
\begin{center}
\begin{table}[ht!]
{\renewcommand{\arraystretch}{1.3}
\begin{tabular}{| r | p{12cm} |}
	\hline
	ID: & 4.E.2\\
    \hline
    Název: & \textbf{Potvrdit rezervaci: Selhání služby odesílající email} \\
    \hline
    Vytvořeno: & Filip Kalous, Daniel Dušek, Anna Popková \\
    \hline
    Popis: & Systém potvrdí/zamítne rezervaci, ale není schopen odeslat informační email. \\
    \hline
    Primární aktéři: & Systém \\
    \hline
    Sekundární aktéři: &  Kancelář \\
    \hline
    Předpoklady: & 
    \begin{minipage}[t]{0.75\textwidth}
    	\begin{enumerate}[nosep,after=\strut]
    		\item Modul systému zodpovědný za odesílání emailu nezvládl odeslat email.
            \item Systém nespadl a potvrzení/zamítnutí rezervace provedl.
    	\end{enumerate}
  	\end{minipage} \\
    \hline
    Následné podmínky: & 
    \begin{minipage}[t]{0.75\textwidth}
    	\begin{enumerate}[nosep,after=\strut]
    		\item Systém neodeslal informační email uživateli.
            \item Systém informoval aktéra Kancelář o neúspěšném odeslání emailu.
            \item Systém poskytl informace aktérovi Kancelář pro maniální odeslání mailu.
    	\end{enumerate}
  	\end{minipage} \\
	\hline
    Akce pro spuštění: & Selhání systému v libovolném místě toku případu \uv{Vytvořit rezervaci}. \\
    \hline
    Tok: & 
    \begin{minipage}[t]{0.75\textwidth}
    	\begin{enumerate}[nosep,after=\strut]
            \item Systém v průběhu vykonávání potvrzení/zamítnutí rezervace generuje obsah emailu pro Zákazníka vytvářejícího rezervaci.
            \item Systém kontaktuje svůj modul pro odeslání emailu a vyžádá si odeslání mailu Zákazníkovi.
            \item Odeslání emailu selže.
            \item Systém zobrazí informační hlášku aktérovi Kancelář informující ho o neúspěchu odesílání emailu. V informační hlášce je obsažen text emailu a adresa na kterou měl být původně zaslán a doporučení provedení tohoto kroku manuálně, mimo systém.
    	\end{enumerate}
  	\end{minipage} \\
    \hline
    Frekvence: & Velice zřídka \\
    \hline

\end{tabular}}
\caption{Vytvořit rezervaci: Selhání služby odesílající email}
\label{table:4}
\end{table}
\end{center}

\newpage
%%%%%%%%%%%%%%%%%%%%%%%%%%
%     USE CASE 5         %
%%%%%%%%%%%%%%%%%%%%%%%%%%
% ZADAT OBJEDNAVKU %
\textbf{Případ užití \uv{Zadat objednávku}}
\begin{table}[ht!]
{\renewcommand{\arraystretch}{1.3}
\begin{tabular}{| r | p{12cm} |}
	\hline
	ID: & 5 \\
    \hline
    Název: & \textbf{Zadat objednávku} \\
    \hline
    Vytvořeno: & Filip Kalous, Daniel Dušek, Anna Popková \\
    \hline
    Popis: & Obslužný personál zadá do systému objednávku a přiřadí ji ke stolu. \\
    \hline
    Primární aktéři: & Oblužný personál\\
    \hline
    Sekundární aktéři: & Systém  \\
    \hline
    Předpoklady: & Zákazník u stolu provedl mimo systém objednávku (například ústně) a obslužný personál zadává do systému informaci o jeho objednávce.  \\
    \hline
    Následné podmínky: & 
    \begin{minipage}[t]{0.75\textwidth}
    	\begin{enumerate}[nosep,after=\strut]
    		\item V systému je vytvořena objednávka.
            \item Vytvořená objednávka je svázána se stolem ke kterému má být doručována.
    	\end{enumerate}
  	\end{minipage} \\
	\hline
    Akce pro spuštění: & Obslužný personál zapne režim \uv{Zadávání objednávky} \\
    \hline
    Hlavní tok: & 
    \begin{minipage}[t]{0.75\textwidth}
    	\begin{enumerate}[nosep,after=\strut]
            \item Aktér Obslužný personál klikne na tlačítko \uv{Zadat objednávku}.
            \item Systém zobrazí uživatelské rozhraní pro zadávání objednávky obsahující (mimo jiné) možnost specifikace stolu, ke kterému se objednávka vztahuje.
            \item Aktér Obslužný personál zadá obsah objednávky a klikne na tlačítko \uv{Zadat}
            \item Systém zkontroluje že Obslužný personál vybral stůl ke kterému se objednávka vztahuje. Tato kontrola proběhne úspěšně.
            \item Systém vytvoří v systému objednávku se všemi zadanými informacemi od obsluhy.
    	\end{enumerate}
  	\end{minipage} \\
    \hline 
    Alternativní toky: & Oblužný personál nezadá objednávající stůl  \\
    \hline
    Výjimky: & 
    \begin{minipage}[t]{0.75\textwidth}
    	\begin{enumerate}[nosep,after=\strut] 
            \item Selhání systému
    	\end{enumerate}
  	\end{minipage} \\
    \hline
    Frekvence: & Často \\
    \hline
    Speciální požadavky: & 
    \begin{minipage}[t]{0.75\textwidth}
    	\begin{enumerate}[nosep,after=\strut]
    		\item Žádné
    	\end{enumerate}
  	\end{minipage} \\
    \hline

\end{tabular}}
\caption{Specifikace případu užití \uv{Zadat objednávku}}
\label{table:1}
\end{table}

\newpage
% ALTERNATIVNI TOK - NEZADANY STUl % 
\begin{table}[ht!]
{\renewcommand{\arraystretch}{1.3}
\begin{tabular}{| r | p{12cm} |}
	\hline
	ID: & 5.1 \\
    \hline
    Název: & \textbf{Zadat objednávku: Oblužný personál nezadá objednávající stůl} \\
    \hline
    Vytvořeno: & Filip Kalous, Daniel Dušek, Anna Popková \\
    \hline
    Popis: & Oblužný personál zadává objednávku bez vyplněného stolu. \\
    \hline
    Primární aktéři: & Obslužný personál, Systém \\
    \hline
    Sekundární aktéři: &  \\
    \hline
    Předpoklady: & Oblužný personál přijal objednávku a zadává ji do systému.  \\
    \hline
    Následné podmínky: & 
	\begin{minipage}[t]{0.75\textwidth}
 		\begin{enumerate}[nosep,after=\strut]
 			\item Systém nevytvoří objednávku u které by nebyl vybraný stůl ke kterému se vztahuje.
            \item Systém navede Obslužný personál aby dokončil zadávání objednávky úspěšně.
 		\end{enumerate}
    \end{minipage} \\
	\hline
    Akce pro spuštění: & Obslužný personál zapne režim \uv{Zadávání objednávky}, zadá objednávku a nevyplní stůl, ke kterému se vztahuje a potvrdí zadávání. \\
    \hline
    Tok: & 
    \begin{minipage}[t]{0.75\textwidth}
    	\begin{enumerate}[nosep,after=\strut]
            \item Aktér Obslužný personál klikne na tlačítko \uv{Zadat objednávku}.
            \item Systém zobrazí uživatelské rozhraní pro zadávání objednávky obsahující (mimo jiné) možnost specifikace stolu, ke kterému se objednávka vztahuje.
            \item Aktér Obslužný personál zadá obsah objednávky bez vyplněného objednávajícího stolu a klikne na tlačítko \uv{Zadat}
            \item Systém zkontroluje že Obslužný personál vybral stůl ke kterému se objednávka vztahuje. Tato kontrola proběhne neúspěšně, protože stůl nebyl vyplněn.
            \item Systém zobrazí chybovou hlášku o chybějícím stolu, ke kterému se objednávka vztahuje a informací pro Obslužný personál, jak tuto informaci doplnil.
            \item Obslužný personál doplní informaci a znovu klikne na tlačítko \uv{Zadat}
            \item Tok dále pokračuje od bodu 4 standardního případu použití.
    	\end{enumerate}
    \end{minipage} \\
    \hline
    Frekvence: & Středně \\
    \hline
    Speciální požadavky: & \\  
        \hline

\end{tabular}}
\caption{Zadat objednávku: Rezervované místo není k dispozici}
\label{table:2}
\end{table}

%%%%%%%%%%%%%%%%%%%%%%%%%%%%%%%
%        USE CASE 6           %
%%%%%%%%%%%%%%%%%%%%%%%%%%%%%%%
\newpage
% Stornovat rezervaci %
\textbf{Případ užití: \uv{Stornovat rezervaci}}
\begin{table}[ht!]
{\renewcommand{\arraystretch}{1.3}
\begin{tabular}{| r | p{12cm} |}
	\hline
	ID: & 6 \\
    \hline
    Název: & \textbf{Stornovat rezervaci} \\
    \hline
    Vytvořeno: & Filip Kalous, Daniel Dušek, Anna Popková \\
    \hline
    Popis: & Zákazník stornuje svou rezervaci \\
    \hline
    Primární aktéři: & Zákazník \\
    \hline
    Sekundární aktéři: & Systém \\
    \hline
    Předpoklady: & Zákazník si vytvořil rezervaci, která je čekající na potvrzení, nebo je již potvrzená. \\
    \hline
    Následné podmínky: & 
    \begin{minipage}[t]{0.75\textwidth}
    	\begin{enumerate}[nosep,after=\strut]
    		\item Rezervace, kterou chtěl Zákazník stornovat je stornována.
            \item Stoly nebo salón, které byly předmětem rezervace jsou opět uvolněny k možné rezervaci.
    	\end{enumerate}
  	\end{minipage} \\
	\hline
    Akce pro spuštění: & Zákazník klikne na odkaz \uv{Stornovat rezervaci} \\
    \hline
    Hlavní tok: & 
    \begin{minipage}[t]{0.75\textwidth}
    	\begin{enumerate}[nosep,after=\strut]
            \item Zákazník klikne na odkaz \uv{Stornovat rezervaci} a je přesměrován na stránku s formulářem pro stornování rezervace.
            \item Zákazník vyplní ve formuláři rezervační klíč a klepne na tlačítko \uv{Stornovat rezervaci}
            \item Zákazník potvrdí, že chce rezervaci opravdu stornovat.
            \item Systém zkontroluje, zda rezervace s poskytnutým klíčem existuje a zda ještě nenastal její čas. Kontrola proběhne úspěšně.
            \item Systém uvolní registrované stoly, popřípadě salónek v době stornované rezervace.
            \item Systém odstraní rezervaci ze systému. 
    	\end{enumerate}
  	\end{minipage} \\
    \hline
    Alternativní toky: & 
    \begin{minipage}[t]{0.75\textwidth}
    	\begin{enumerate}[nosep,after=\strut]
            \item Zákazník vyplní špatný rezervační klíč.
            \item Zákazník vyplní rezervační klíče rezervace, která již započala.
    	\end{enumerate}
  	\end{minipage} \\
    \hline
    Výjimky: & 
    \begin{minipage}[t]{0.75\textwidth}
    	\begin{enumerate}[nosep,after=\strut]
    		\item Selhání systému
    	\end{enumerate}
  	\end{minipage} \\
    \hline
    Frekvence: & Středně často \\
    \hline
    Speciální požadavky: & 
    \begin{minipage}[t]{0.75\textwidth}
    	\begin{enumerate}[nosep,after=\strut]
    		\item Žádné
    	\end{enumerate}
  	\end{minipage} \\
    \hline

\end{tabular}}
\caption{Specifikace případu užití \uv{Stornovat rezervaci}}
\label{table:1}
\end{table}

\newpage
% ALTERNATIVNI TOK - Špatný rezervační klíč % 
\textbf{Alternativní tok: Špatný rezervační klíč}
\begin{table}[ht!]
{\renewcommand{\arraystretch}{1.3}
\begin{tabular}{| r | p{12cm} |}
	\hline
	ID: & 6.1 \\
    \hline
    Název: & \textbf{Stornovat rezervaci: Zákazník vyplní špatný rezervační klíč}. \\
    \hline
    Vytvořeno: & Filip Kalous, Daniel Dušek, Anna Popková \\
    \hline
    Popis: & Zákazník stornuje svou rezervaci a používá špatný rezervační klíč. \\
    \hline
    Primární aktéři: & Zákazník \\
    \hline
    Sekundární aktéři: &  Systém \\
    \hline
    Předpoklady: & Zákazník se pokouší využít funkce stornování rezervace.  \\
    \hline
    Následné podmínky: & 
	\begin{minipage}[t]{0.75\textwidth}
 		\begin{enumerate}[nosep,after=\strut]
 			\item Systém neodrezervuje žádné stoly ani salón.
            \item Systém informuje Zákazníka o důvodu neúspěchu zrušení rezervace.
 		\end{enumerate}
    \end{minipage} \\
	\hline
    Akce pro spuštění: & Zákazník odešle formulář stornující rezervaci. \\
    \hline
    Tok: & 
    \begin{minipage}[t]{0.75\textwidth}
    	\begin{enumerate}[nosep,after=\strut]
            \item Zákazník klikne na odkaz \uv{Stornovat rezervaci} a je přesměrován na stránku s formulářem pro stornování rezervace.
            \item Zákazník vyplní ve formuláři rezervační klíč a klepne na tlačítko \uv{Stornovat rezervaci}
            \item Zákazník potvrdí, že chce rezervaci opravdu stornovat.
            \item Systém zkontroluje, zda rezervace s poskytnutým klíčem existuje a zda ještě nenastal její čas. Kontrola proběhne neúspěšně, protože odpovídající rezervace se zadaným klíčem neexistuje.
            \item Systém se nepokouší uvolňovat žádné stoly ani salónek a zobrazí uživateli chybovou hlášku o špatném rezervačním klíči.
    	\end{enumerate}
    \end{minipage} \\
    \hline
    Frekvence: & Málo často \\
    \hline
    Speciální požadavky: & \\  
        \hline

\end{tabular}}
\caption{Stornovat rezervaci: Zákazník vyplní špatný rezervační klíč}
\label{table:2}
\end{table}

% ALTERNATIVNI TOK - Zákazník ruší rezervaci, která již probíhá % 
\newpage
\textbf{Alternativní tok: Zákazník ruší rezervaci, která již probíhá}
\begin{table}[ht!]
{\renewcommand{\arraystretch}{1.3}
\begin{tabular}{| r | p{12cm} |}
	\hline
	ID: & 6.2 \\
    \hline
    Název: & \textbf{Stornovat rezervaci: Zákazník vyplní rezervační klíče rezervace, která již započala.}. \\
    \hline
    Vytvořeno: & Filip Kalous, Daniel Dušek, Anna Popková \\
    \hline
    Popis: & Zákazník stornuje svou rezervaci, avšak čas aktivace rezervace již nastal. \\
    \hline
    Primární aktéři: & Zákazník \\
    \hline
    Sekundární aktéři: &  Systém, Kancelář \\
    \hline
    Předpoklady: & Zákazník se pokouší využít funkce stornování rezervace, ale jeho rezervace již začala.  \\
    \hline
    Následné podmínky: & 
	\begin{minipage}[t]{0.75\textwidth}
 		\begin{enumerate}[nosep,after=\strut]
        	\item Systém uvolní rezervované stoly, nebo salónek.
 			\item Systém zruší rezervaci.
            \item Systém informuje kancelář o tom, že probíhající rezervace byla zrušena.
 		\end{enumerate}
    \end{minipage} \\
	\hline
    Akce pro spuštění: & Zákazník odešle formulář stornující rezervaci. \\
    \hline
    Tok: & 
    \begin{minipage}[t]{0.75\textwidth}
    	\begin{enumerate}[nosep,after=\strut]
            \item Zákazník klikne na odkaz \uv{Stornovat rezervaci} a je přesměrován na stránku s formulářem pro stornování rezervace.
            \item Zákazník vyplní ve formuláři rezervační klíč a klepne na tlačítko \uv{Stornovat rezervaci}
            \item Zákazník potvrdí, že chce rezervaci opravdu stornovat.
            \item Systém zkontroluje, zda rezervace s poskytnutým klíčem existuje a zda ještě nenastal její čas. Kontrola proběhne neúspěšně, protože čas aktivace rezervace již nastal.
            \item Systém uvolní registrované stoly, popřípadě salónek v době stornované rezervace.
            \item Systém informuje Kancelář, že probíhající rezervace byla zrušena.
            \item Systém odstraní rezervaci ze systému. 
    	\end{enumerate}
    \end{minipage} \\
    \hline
    Frekvence: & Málo často \\
    \hline
    Speciální požadavky: & \\  
        \hline

\end{tabular}}
\caption{Stornovat rezervaci: Zákazník stornuje svou rezervaci, avšak čas aktivace rezervace již nastal.}
\label{table:2}
\end{table}

% JESTE VYJIMKY? %
\textbf{Výjimka 6.E.1 \textit{Selhání systému} je stejná jako výjimka 1.E.3.}
\newpage


\end{document}